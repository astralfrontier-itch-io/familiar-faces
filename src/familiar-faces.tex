\environment env_book

\starttext

\title{Familiar Faces}

\subtitle{\goto{CC-BY 4.0}[url(https://creativecommons.org/licenses/by/4.0/)] by \goto{Astral Frontier}[url(https://astralfrontier.itch.io)]}

\startcolumns[TwoColumn]

\section{What Is This?}

This is a tool to create interesting supporting characters for games or other stories.
It focuses on emotion, motivation, and conflicts.

There are three steps:

\startitemize[n,packed]
\item Anchor the Character
\item Create conflicts
\item Interview the Character
\stopitemize

\column

\section{Principles}

\startitemize[4]
\item The audience cares about a character because of what they need and what they're doing about it
\item Start with the most fundamental and universal, and work out to the most specific
\item It's better to start with the wrong answer than an empty page - we can always rewrite what we have
\item Supporting characters' conflicts and drives should relate to the protagonists' stories
\stopitemize

\stopcolumns

\page

\input table

\section{Anchor the Character}

\startcolumns[TwoColumn]

Why did you feel the need to create this character? What role will they play in your story?
This is an character's {\gameterm reason for being}.

Examples:

\startSmallNarrower[right]

\startitemize[4]
\item an antagonist, obstacle, or lover
\item a citizen: king, town guard, or sage
\item a typical resident of a specific location, such as a teacher for a school or a merchant for a marketplace
\item a character with a particular nature: funny, serious, alien, etc.
\item a representative for a social class, a group, or a belief
\item a source of information, expertise, or resources needed to solve a problem
\stopitemize

\stopSmallNarrower

\column

Next, choose an emotion from the table above.
Roll a d8 and read down, then roll another d8 and read right.

Finally, relate that emotion back to their reason for being. An easy method is to write a sentence like:

\startitemize[4]
\item {\bf This character is (reason for being) because they feel (emotion)}
\item {\bf This character feels (emotion) because they're (reason for being)}
\stopitemize

The resulting sentence is the character's {\gameterm anchor}. The character's actions and reactions should be driven by this anchor.

\stopcolumns

\page

\input table

\section{Create Conflicts}

Conflicts are the heart of what makes a character interesting:
they want something, it's hard for them to get it, and
what they try to do about that is what makes a story.

How many conflicts should you assign to a character?
Minor characters might have zero or one conflicts.
Major characters can have two or three.

Characters can have two types of conflicts: internal and external.
When choosing the focal points of conflicts, 
look at the protagonists' own backgrounds and interests so that they're relatable to the protagonists.

\startcolumns[TwoColumn]

\subsection{Internal Conflicts}

Internal conflicts are when a character struggles with their own
emotions, motivations, and desires.
They might involve other people but they are not conflicts with other people.

\startitemize[4]
\item Choose an emotion from the table above
\item Name the source or object of that emotion
\item {\bf Anchor} the emotion to the source or object
\stopitemize

\subsection{External Conflicts}

External conflicts are disagreements, quarrels, or other struggles
with another character.
The other character may or may not feel the conflict as well.

\startitemize[4]
\item Choose an emotion from the table above
\item Pick another character or a group as a whole
\item Name a shared responsibility, interest, or other activity
\item {\bf Anchor} the emotion to the shared element
\stopitemize

\stopcolumns

{\bf Anchors} relate the emotion and some story element ("I feel X because of this" vs. "I'm conflicted because of this because I feel X").

\page

% Interview the Character

\section{Interview the Character}

Journalists ask questions: who, what, when, where, how and why.
Once you establish your character's anchor and conflicts,
pose these questions to your character.
Follow up on the answers with fresh, exploratory questions.

The best interview questions will twist the knife by uncovering
uncomfortable truths.
Where possible, ask questions that would be hard for the character to answer honestly.

Try to answer at least two questions.
The more you answer, the more detailed the character will be.
Five questions will give you a character with a high level of detail.

Every player in a game should be given an opportunity to answer
interview questions for a supporting character they'll interact with.

\subsection{Example Interview Questions}

\startFullPageNarrower[left,right]

\startitemize[1]
\item How did you arrive at your position (king, goatherd, etc.)? How far will you go to escape it?
\item Who helped you get where you are (this place, in life)? How? What did it cost them?
\item Are you doing what you want to be doing? If not, what would you rather do?
\item Are you moving toward something in your life, or away from something? What?
\item Who or what do you respect? Why? Have they earned it?
\item How are current events affecting you? How are you reacting to them?
\item {\it For any conflict...}: how did this start? How will it end?
\item {\it Admiration, Trust, ...}: have your feelings been betrayed or reinforced over time?
\item {\it Terror, Fear, ...}: what helps you maintain self-control despite your fear?
\item {\it Ecstacy, Joy, ...}: how do you spread your happiness to others? Who doesn't accept it?
\item {\it Amazement, Surprise, ...}: what did you think was going to happen instead of this?
\item {\it Grief, Sadness, ...}: who or what helps you through it?
\item {\it Loathing, Disgust, ...}: who or what helps you find peace?
\item {\it Rage, Anger, ...}: who or what helps you calm down?
\item {\it Vigilance, Anticipation, ...}: what are the moments when you can relax?
\stopitemize

\stopFullPageNarrower

\page

% Example Characters

\section{Example Characters}

We want to create characters for a city-based campaign,
where our heroes will face both political maneuvering and back-alley brawls.
All we know is that all three characters want to change the city somehow.

\subsection{The Noble}

Let's start with an upper-class character.
To anchor our character, we roll for an emotion and get 3 and 7.
This gives us {\bf Optimism}.
They're optimistic because they have high social status.

We'll give them two external conflicts.
Rolling, we get 2 and 6: {\bf Apprehension}.
Let's say this is targeted at another well-to-do person, who we'll call the Magnate.
Their shared interest is the city's economy.
Thus, the conflict is that the Magnate's recent economic activity worries the Noble.

The second conflict is internal.
We get a 4 and 8, {\bf Disapproval}.
We'll say it's about the people of the city.
The Noble thinks that the people aren't doing enough to help themselves.

A logical interview question is:
"if you think people aren't doing enough (Apprehension and Disapproval),
why do you think things are going to be okay (Optimism)?"
The Noble's answer comes from a place of privilege.
They think that just because they're an aristocrat,
that they can solve this with money and influence.
Events shall see if that belief is validated.

Next we can ask, "how are you trying to help?"
Their answer is "giving money to what seem to be worthy causes".

Taking the advice to twist the knife and uncover truths,
let's also ask, "how do you know what are the worthy causes?"
We'll say the Noble relies both on the advice of friends and their own intuition,
but doesn't follow up personally on how well their money is spent.

\subsection{The Demagogue}

Let's descend the ladder and look for a rabble-rouser or influencer
who occupies a lower or middle class.

Our anchoring emotional roll is 7 and 8, giving us {\bf Aggression} - very suitable.
It would be easy to say their feelings led to their position,
but what if we reverse it?
They are Aggressive because they are a Demagogue.

They have an external conflict and an internal conflict.
The emotion roll for the external conflict is 6 and 8, giving us {\bf Contempt}.
An easy target for their contempt is the Noble we just created.
Their shared interest is the city's well-being,
and the Demagogue thinks the Noble's frivolous spending is hurting
far more than it could help.

For the internal conflict, we roll 3 and 3, getting {\bf Joy}.
Let's have fun and make this indirectly about the Noble again.
The Demagogue really liked the bits of the high life they experienced from time to time.
The conflict is whether it's okay to like fancy parties and dinners
when you're supposed to be fighting for the city.

For the interview, let's start with the anchor.
"Why are you aggressive because you're a demagogue"?
The character became an advocate for a consortium of merchants,
and only discovered how bad things really were after investigating the issues in that role.

As a followup, we can ask "are you yourself a merchant?" Sure, makes sense, let's go with that.

Finally, looking back at Aggression, we can ask, "what keeps you civil when you'd rather not be?"
The answer is that the Demagogue has family - maybe a spouse, maybe children - who
would lose their financial support if something escalated.
Thus, the Demagogue wants to start something, feeling there's little alternative left,
but is constrained by the need to stay out of trouble.

\subsection{The Schemer}

Finally, let's look at someone tied to the systems of power, but not part of them.
This character will be a plotter and manipulator behind the scenes.

Their anchoring emotion is {\bf Sadness}, with a roll of 5 and 3.
We can say that they took up this life because of that sadness.

They have an external conflict, based on {\bf Admiration}.
We'll say that this is directed at a character we introduced in the Noble's creation: the Magnate.
The shared interest is in personal wealth and power.
So what's the issue? We'll say that the Schemer likes how the Magnate gets things done,
but that their own plans need the Magnate to go down as well.

They have an internal conflict. On a 6 and 1, this is based on {\bf Loathing}.
We'll tie this to the overall issue of politics and wealth in the city.

It feels like the Schemer isn't at peace with who and what they are.
Why not? Let's ask: "why do you feel sadness and loathing? Why do these motivate you?"
We'll say that they lost someone special to them, like a spouse or family member,
because of the situation in the city.
Perhaps medicine was too expensive, for example.

"Why do you feel loathing about this situation?"
The answer is because that lost special person would not want the Schemer
to be doing what they're doing, and yet this seems to be the only way for
it to never happen again.
The Schemer has become a tragic figure to avert further tragedy, at least in their own eyes.

Let's twist the knife.
"Do you want to bring it all down, or do you want to become one of them?"
The Schemer has no good answer for that, and this is probably the root
of their external conflict with the Magnate.

\page

% Faces and Fractured Mirrors

\section{Faces and Fractured Mirrors}

Many characters in fiction stand in for something larger, such as a government, a religion, or an ideology.
We learn about those larger things by watching the character.
For example, in "Star Wars", we see two sides of the Empire through Tarkin and Darth Vader.
Tarkin represents the banal brutality of a soldier serving under a fascist regime.
Vader represents the fanatical, emotionally centered Sith religion at the heart of the Empire.
We can say that they are the {\it faces} of such things in our story.

At the same time, such faces are imperfect reflections, because everyone is an individual.
Characters with strong loyalty can disagree with their parent entity about something.
Other times, characters may be unable or unwilling to comply with the demands of that larger force.
These conflicts between individual and group can tell us about the world as well.
For example, in "Inception", we learn about the process of extraction by meeting Dom Cobb and his crew.
But because of his guilt over his wife's fate, Dom can't perform a key step in the process: architecting dreams.
Thus, we learn about architecting as he explains the process to someone else.

When you are creating such face characters, ask interview questions that tease out these elements.
How does this character exemplify what they stand in for? How do they contrast against it?

\section{Other Tools}

Many excellent tools can help you further define your character,
such as \goto{Fantasy Name Generators}[url(https://www.fantasynamegenerators.com/)]
and \goto{Character Appearance Generator}[url(https://www.rangen.co.uk/chars/appgen.php)].

\section{Acknowledgements}

This document was created via \goto{ConTeXt}[url(https://wiki.contextgarden.net/Main_Page)], licensed under the GNU GPL v2.

Plutchik's Wheel of Emotion was taken from \goto{File:Plutchik-wheel.svg}[url(https://commons.wikimedia.org/wiki/File:Plutchik-wheel.svg)]
and is in the public domain.

The header font is Abrakadaboom, designed by \goto{Kurasan}[url(https://www.fontspace.com/abrakadaboom-font-f124405)].
The body font is \goto{EB Garmond}[url(https://fonts.google.com/specimen/EB+Garamond/license)], under the SIL Open Font license, version 11.

\page

% Wheel of Emotion

% "none" means suppress the title
\placefigure
    [here,none]
    [fig:plutchik]
    {Plutchik's wheel of emotion}
    {\externalfigure[plutchik][type=png]}

Robert Plutchik's Wheel of Emotions shows how emotions relate to each other,
both in intensity (the strongest emotions are closest to the center)
and in nature (an emotion's inverse is on the opposite side of the wheel from it).

\page

\stoptext