\environment env_book

\starttext

\title{Familiar Faces}

\subtitle{\goto{CC-BY 4.0}[url(https://creativecommons.org/licenses/by/4.0/)] by \goto{Astral Frontier}[url(https://astralfrontier.itch.io)]}

\startcolumns[TwoColumn]

\section{What Is This?}

This is a tool to create interesting supporting characters for games or other stories.
It focuses on emotion, motivation, and conflicts.

There are three steps:

\startitemize[n,packed]
\item Anchor the Character
\item Create conflicts
\item Interview the Character
\stopitemize

Whenever you are asked to "roll for an emotion", you'll roll 2d8 and consult the table below.
Read down for the value of the first d8, then across for the second.

\column

\section{Principles}

\startitemize[4]
\item The audience cares about a character because of what they need and what they're doing about it
\item Small, engaging details create a spark of humanity
\item It's better to start with the wrong answer than an empty page - we can always rewrite what we have
\item Supporting characters' conflicts and drives should relate to the protagonists' stories
\stopitemize

\stopcolumns

\input table

\page

\section{The Wheel of Emotions}

% "none" means suppress the title
\placefigure
    [here,none]
    [fig:plutchik]
    {Plutchik's wheel of emotion}
    {\externalfigure[plutchik][type=png]}

\startalignment[middle]

Robert Plutchik's Wheel of Emotions

\stopalignment

\page

\section{Anchor the Character}

\startcolumns[TwoColumn]

Why did you feel the need to create this character? What role will they play in your story?
This is an character's {\gameterm reason for being}.

Examples:

\startSmallNarrower[right]

\startitemize[4]
\item an antagonist, obstacle, or lover
\item a citizen: king, town guard, or sage
\item a typical resident of a specific location, such as a teacher for a school or a merchant for a marketplace
\item a character with a particular nature: funny, serious, alien, etc.
\item a representative for a social class, a group, or a belief
\item a source of information, expertise, or resources needed to solve a problem
\stopitemize

\stopSmallNarrower

\column

Next, choose an emotion from the table above.
Roll a d8 and read down, then roll another d8 and read right.

Finally, relate that emotion back to their reason for being. An easy method is to write a sentence like:

\startitemize[4]
\item This character is (reason for being) because they feel (emotion)
\item This character feels (emotion) because they're (reason for being)
\stopitemize

The resulting sentence is the character's {\gameterm anchor}. The character's actions and reactions should be driven by this anchor.

\stopcolumns

\section{Create Conflicts}

Conflicts are the heart of what makes a character interesting:
they want something, it's hard for them to get it, and
what they try to do about that is what makes a story.

How many conflicts should you assign to a character?
If a character doesn't have immediate story importance, you can get away with no conflicts.
Minor characters might have one conflict.
Major characters can have two or three.

Characters can have two types of conflicts: internal and external.

\startcolumns[TwoColumn]

\subsection{Internal Conflicts}

To create an internal conflict, choose an emotion from the table above.

Then, name the source or object of that emotion.
The easiest way is to pick one or more PCs, and look at their backstories, goals, or affiliations.
You can also pick an issue that's active in the campaign.
Relate the emotion to this thing, the way you created an anchor.

\subsection{External Conflicts}

To create an external conflict, choose an emotion from the table above.
Next, pick another character or a group as a whole.
Then, name a shared responsibility, interest, or other activity.
Finally, anchor the conflict by explaining how that shared element
sparks the chosen emotion in the character.

\stopcolumns

\section{Interview the Character}

Journalists ask questions: who, what, when, where, how and why.
Once you establish your character's anchor and conflicts,
pose these questions to your character.
Follow up on the answers with fresh, exploratory questions.

Try to answer at least two questions.
The more you answer, the more detailed the character will be.
Five questions will give you a character with a high level of detail.

Every player in a game should be given an opportunity to answer
interview questions for a supporting character they'll interact with.

\subsection{Example Interview Questions}

Some of these questions mention "the issue". This is a shorthand for whatever story issues are important at the moment.
For example, if an enemy army is invading, "the issue" means "the invasion".

\startFullPageNarrower[left,right]

\startitemize[1]
\item How did you arrive at your position (king, goatherd, etc.)?
\item Who helped you get here?
\item Who doesn't like you being here?
\item Do you have any family, close friends, etc. in the area?
\item Are things going well for you, or badly? If badly, what happened?
\item Are you doing what you want to be doing? If not, what would you rather do?
\item How is the issue affecting you?
\item How have you responded to the issue?
\item Who or what do you respect?
\stopitemize

\stopFullPageNarrower

\page

\section{Example Characters}

TODO

\page

\section{Acknowledgements}

Plutchik's Wheel of Emotion was taken from \goto{File:Plutchik-wheel.svg}[url(https://commons.wikimedia.org/wiki/File:Plutchik-wheel.svg)]

TODO: font uses

TODO: playtest credits

\stoptext